\section{Mathematical Preliminaries and Error Analysis}
\subsection{Round-off Errors and Computer Arithmetic}
\subsubsection{Binary machine numbers}
\begin{defn}{舍入误差}
舍入误差形成原因:进行有限位的运算(finite digits arithmetic)\par
其中,IEEE:754-2008规定二进制机器数(Binary machine numbers)中浮点数(floating-point)存储规范如下:
    \begin{align*}
    (-1)^S2^{c-1023}(1+f) \\
    \begin{cases}
        S:\, 0/1 & sign part \\
        c:\, 11\,digits & \text{exponential part.} \\
        f:\, 52\,digits & \text{mantissa part} \\
    \end{cases}
    % \square\square\cdots\square
    \end{align*}
\end{defn}\par
由于实数的稠密性,可知找不到比某一个数大的最小的数或小的最大的数,但是在计算机中可以找到,所以计算机不能表示所有的数。

\subsubsection{Decimal machine numbers}
$\pm 0.d_1d_2\cdots d_n \times 10^n$其中$1\leq d_1\leq 9$,$0\leq d_i\leq 9$,$\forall i\geq 2$。如果记真实的数为$y$,其浮点数表示为$fl(y)$。\par
当存在$y=0.d_1d_2\cdots d_kd_{k+1}\cdots\times 10^n$,其浮点数表示有如下两种方式:
\begin{enumerate}
    \item Chopping: chop off digits, say $d_{k+1}d_{k+2}\cdots$.
    \item Rounding: $y+5\times 10^{n-(k+1)}$, then chopping.
\end{enumerate}
\begin{example}
$\pi=3.14159265\cdots$,取5位。\par
    \begin{itemize}
      \item Chopping: $fl(y)=0.31415\times 10^1$.
      \item Rounding: $fl(y)=0.31416\times 10^1$.
    \end{itemize}
\end{example}\par
\begin{defn}{ }
Suppose $p^\star$ is an approximation of p.
\begin{align*}
    \begin{cases}
        absolute\,error\,&=\,\left|p^\star-p\right| \\
        relative\,error\,&=\,\frac{\left|p^\star-p\right|}{p} \\
    \end{cases}
\end{align*}
\end{defn}

\begin{defn}{有效数字(Significant digits)}
$p^\star$ is said to approximate p with \emph{t} significant digits. If t is the \emph{largest nonnegative} integer, s.t. $$\frac{\left|p-p^\star\right|}{p}\leq 5\times 10^{-t}$$.
\end{defn}\par
Chopping floating:\par
$$y=0.d_1\cdots d_kd_{k+1}\cdots\times 10^n$$
$$fl(y) = 0.d_1\cdots d_k\times 10^n$$\par
Chopping:(其有效位数至少为k-1)
$$
\frac{\left|fl(y)-y\right|}{\left|y\right|} = \frac{0.0\cdots 0d_{k+1}\cdots\times 10^n}{0.d_1\cdots d_kd_{k+1}\cdots\times 10^n} \leq 10^{1-k}
$$\par
Rounding:(其有效位数至少为k)
$$
\frac{\left|fl(y)-y\right|}{\left|y\right|} \leq \frac{0.0\cdots 1d_{k+1}\cdots\times 10^n}{0.d_1\cdots d_kd_{k+1}\cdots\times 10^n} \leq 10^{-k}
$$

\subsubsection{Machine Operators}
记计算机的加减乘除为$\oplus\ominus\otimes\oslash$,于是有
$$x\op y\,=\,fl(fl(x)\op fl(y))$$\par
Four cases to avoid:
\begin{enumerate}
    \item 两个十分接近的数(two nearly equal)。
    \item 分子远大于分母(numerator >> denominator)。
    \item 避免大数吃掉小数。
\end{enumerate}

\subsubsection{Nested method(秦九韶算法)}
\IncMargin{1em}
\begin{algorithm}
\SetKwInOut{Input}{input}\SetKwInOut{Output}{output}
\Input{$a_0,a_1,\cdots,a_n(given)$; $x$}
\Output{$P_n(x)$}
\BlankLine
    $S_n\leftarrow a_n$\;
    \For{$k\leftarrow n-2$ \KwTo $0$}{
        $S_k\leftarrow xS_{k+1}+a_k$\;
    }
    $P_n(x)\leftarrow S_0$\;
\end{algorithm}
\begin{python}
def nested(poly:list=[1], x:float=0.0)->float:
    """
    Horner nested polynomial calculation.

    Args:
        poly: List, store the coefficient of the polynomial.
        x: Float, specify the variable in the polynomial.

    Returns:
        Float, result.

    Raises:
        If `poly' is empty, raise IndexError.
        If type(args) does not correspond, raise TypeError.
    """
    result = poly[0]
    for i in range(1, poly.__len__()):
        result = x*result + poly[i]
    return result
\end{python}

\subsubsection{Convergence(收敛性)}
Stable: small change in initial data and the error is small.\par
若$E_0$为初始值误差,$E_n$为$n$步的误差,
\begin{itemize}
    \item $E_n\thickapprox C$(不依赖$n$),称之为线性。
    \item $E_n\thickapprox C^nE_0$则可由$C$的取值判断是否稳定。
\end{itemize}
\begin{defn}{Rates of Convergence}
当$n\rightarrow\infty$,$\alpha_n\rightarrow\alpha$,$\beta_n\rightarrow 0$,其中$\left|\alpha_n-\alpha\right|\leq k\left|\beta_n\right|$(与$n$的取值无关),则称$\alpha_n$是以$\beta_n$的速度收敛到$\alpha$的。
\[
\alpha_n\,=\,\alpha+o(\beta_n).
\]
\end{defn}
